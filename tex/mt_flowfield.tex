% vim: set spell:

\section{Flow field pathfinding}

Flow field pathfinding(Búsqueda de camino por campos de flujo) es una técnica
sencilla especialmente útil cuando gran cantidad de agentes comparten el mismo
conjunto de metas. Dado un grafo, se calcula el costo desde cada nodo a la meta
mas cercana creado un nuevo grafo. Con este nuevo grafo se calcula un tercer
grafo de gradientes, simplemente calculando para cada nodo la dirección de aquel
vecino con menor costo hasta la meta. De esta manera, los agentes solo necesitan
conocer en que nodo se encuentra, y seguir el gradiente indicado en este.

En el caso de estudio de esta tesis veremos su aplicación a video juegos, en
especial en aquellos de estrategia en tiempo real, ya que estos normalmente
poseen una gran cantidad de agentes(normalmente denominados unidades) con metas
compartidas. Usualmente las unidades se movilizan sobre un mapa con obstáculos
dinámicos(incluyendo otras unidades), estáticos y otros temporales pero que no
cambian de posición o lo hacen muy esporádicamente. El costo definido en el
párrafo anterior puede tener distintas interpretaciones. Por un lado puede ser
simplemente la distancia entre un punto a otro, pero también puede variar
dependiendo de múltiples factores, como zonas que es deseable evitar o la
densidad de unidades(lo que ayuda a evitar la congestión).

En 'Continuum Crowds'\cite{CC06} se presenta el problema de búsqueda de caminos
como un problema de minimización de energía de partículas. Proveen un robusto
marco teórico donde se busca el camino que minimice $ \sigma(P) $, definida
como:

\begin{equation} \sigma(P) = \alpha \int_P 1 ds + \beta \int_P 1 dt + \gamma
\int_P g(x) dt \label{eqCost} \end{equation}

Donde $dt$ indica que la integral se toma con respecto del tiempo, y $ds$ con
respecto a la distancia. $ g(x) $ es una función que determina el grado de
disconformidad que genera el punto $ x $. Las constantes $\alpha$, $\beta$ y
$\gamma$ pueden variar entre grupos unidades, y representan que importancia le
asigna esta a la longitud del camino, el tiempo que se tarde en recorrerlo y
evitar las zonas de disconformidad.

Si $ f $ es la rapidez del agente, la ecuación puede ser re escrita como $
\sigma(P) = \int_P C(x) ds $ , donde:

\begin{equation} C(x) = \frac{ \alpha f + \beta + \gamma g(x) } { f }
\label{costoCamino}\end{equation} 

Si para cada punto $x$ elegimos el camino que minimice $\sigma(P)$ podemos
definir una nueva función $\phi(x)$, que indique para cada punto su distancia
mínima a la meta. Para un agente en la posición $x$ la estrategia optima sera
moverse en dirección contraria al gradiente de $\phi(x)$. Utilizando la ecuación
\ref{costoCamino}, podemos definir $\phi(x)$ como $\phi(M)=0$ para todo $M$ que
sea meta, y para cualquier otra posición debe satisfacer la ecuación eikonal:

\begin{equation} || \nabla\phi(x) || = C(x)
\label{defphi}\end{equation} 
