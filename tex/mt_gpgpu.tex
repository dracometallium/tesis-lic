% vim: set spell:

\chapter{Características principales de la \emph{gpu} como procesador de
propósito general}

La diferencia principal entre la \emph{gpu} y la \emph{cpu} se encuentra en su
capacidad de procesamiento paralelo. Hoy en día es fácil encontrar computadoras
personales de escritorio y portátiles de cuatro u ocho núcleos, esto le permite
ejecutar un thread por núcleo con gran facilidad. Una \emph{gpu} moderna puede
ejecutar 1024 threads con la misma facilidad, siempre que todos ejecuten la
misma función. Su modelo de ejecución es similar a la arquitectura
\emph{simd}(\emph{Single Instruction, Multiple Data}), pero dos threads pueden
ejecutar distintos caminos de datos siempre que se encuentren en distintos
\emph{wavefronts}. Por esto \emph{Nvidia} distingue a este modelo llamándolo
\emph{simt}(\emph{Single Instruction, Multiple Thread}).

Las funciones ejecutadas por la \emph{gpu} son llamadas \emph{kernels}. La
diferencia entre un \emph{kernel} y una función convencional, es que los
\emph{kernels} se ejecutan sobre un dominio de indices multidimensional. Cada
uno de los threads que ejecutan el \emph{kernel} son llamados \emph{workitems}.
El conjunto de \emph{workitems} que se ejecutan juntos es llamado
\emph{wavefront}. Si los caminos de datos de dos \emph{workitems} en un mismo
\emph{wavefront} recorren distintas ramas de una estructura de control, cada una
de estas se ejecuta en serie. Es por esto que siempre se debe intentar que los
\emph{workitems} de un mismo \emph{wavefront} recorran el mismo camino de datos.
Los dispositivos actuales tienen \emph{wavefrons} de 32 o 64 \emph{workitems}.
Los \emph{workitems} son lanzados en forma secuencial y son asignados a los
\emph{wavefronts} de forma ordenada, por lo tanto, si el dispositivo tiene
\emph{wavefronts} de tamaño $N$, el \emph{wavefront} $K$ estará formado por los
\emph{workitems} $(K*N)$ al $(K*(N+1)-1)$. Los \emph{wavefronts} se agrupan en
\emph{workgroups}. Los \emph{workitems} de un mismo \emph{workgroup} pueden
sincronizar su ejecución a través de barreras y pueden compartir información por
medio de una memoria local. \emph{workitems} de \emph{workgroups} diferentes
solo pueden compartir datos utilizando la memoria global, que es mucho mas lenta
que la local, y no pueden sincronizar su ejecución.

Su organización jerárquica de ejecución se refleja en su arquitectura. En la
cima de la jerarquía nos encontramos con el \emph{host} y el \emph{device}. El
\emph{host} es el dispositivo que ejecuta el programa principal, el
\emph{device} es la \emph{gpu} en si misma (se debe aclarar que \emph{OpenCL} permite
que el \emph{device} sea otro tipo de dispositivo, incluyendo, pero no limitado
a, un \emph{cpu}). El \emph{host} es el encargado de coordinar el funcionamiento
del \emph{gpu}, proveyéndole de datos iniciales, cargando el \emph{kernel} a
ejecutar y coordinando su ejecución. El \emph{device} tiene dos tipos de memoria
propia, la memoria global, que puede ser accedida por todos los
\emph{workitems}, pero es de acceso lento, y la memoria constante que puede ser
leída pero no modificada por todos los \emph{workitems}. Ambas memorias deben
ser reservadas por el \emph{host} y en el caso de la constante solo puede ser
iniciada por este. La transferencia de datos entre la memoria global del
\emph{device} y la memoria del \emph{host} es muy lenta, y es una de las
principales fuentes de overhead en los programas que utilizan \emph{OpenCL}.

El \emph{device} esta compuesto por una o mas \emph{computing units}, es en
estas donde se encuentra la memoria local. Esta memoria es mas rápida que la
memoria global.  Todos los \emph{workitems} de un \emph{workgroup} ejecutan en
la misma \emph{computing unit}, pero una \emph{computing unit} puede ejecutar
distintos \emph{workgroups}. Finalmente, cada \emph{computing unit} esta formada
por uno o varios \emph{processing} \emph{elements}, es en estos que se ejecutan
los \emph{workitems}. Poseen registros muy rápidos pero pequeños que son
utilizados para las variables locales de los \emph{workitems}. Normalmente cada
\emph{computing unit} poseen mas de una ALU, y en cada ciclo de reloj ejecuta
varios \emph{workitems} de un mismo \emph{wavefront}.
