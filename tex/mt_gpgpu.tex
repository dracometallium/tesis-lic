% vim: set spell:

\chapter{Características principales de la GPU como procesador de propósito
general}

La diferencia principal entre la GPU y la CPU se encuentra en su capacidad de
procesamiento paralelo. Hoy en día es fácil encontrar computadoras personales de
escritorio y portátiles de cuatro u ocho núcleos, esto le permite ejecutar un
thread por núcleo con gran facilidad. Una GPU moderna puede ejecutar 1024
threads con la misma facilidad, siempre que todos ejecuten la misma función. Su
modelo de ejecución es similar a la arquitectura SIMD(Single Instruction,
Multiple Data), pero dado que dos threads pueden ejecutar distintos caminos de
datos siempre que se encuentren en distintos WAVEFRONTS. Por esto nvidia llama a
este modelo SIMT(Single Instruction, Multiple Thread).

Las funciones ejecutadas por la GPU son llamadas KERNELS. La diferencia entre un
KERNEL y una función convencional, es que los KERNELS se ejecutan sobre un
dominio de indices multidimensional. Cada uno de los threads que ejecutan los
KERNELS son llamados WORKITEMS. El conjunto de WORKITEMS que se ejecuta en
conjunto es llamado WAVEFRONT. Si los caminos de datos de  WORKITEMS en un mismo
WAVEFRONT recorren distintas ramas de una estructura de control, cada una de
estas se ejecuta en serie. Es por esto que siempre se debe intentar que los
WORKITEMS de un mismo WAVEFRONT recorran el mismo camino de datos. Los
dispositivos actuales tienen WAVEFRONST de 32 0 64 WORKITEMS. Los WORKITEMS son
lanzados en forma secuencial y son asignados a los WAVEFRONTS de forma ordenada.
Por lo tanto, si el dispositivo tiene WAVEFRONTS de tamaño N, el WAVEFRONT K
estará formado por los WORKITEMS (K*N) al ((K*N+1)-1). Los WAVEFRONTS se agrupan
en WORKGROUPS. Los WORKITEMS de un mismo WORKGROUP pueden sincronizar su
ejecución a través de barreras y pueden compartir información por medio de una
memoria local. WORKITEMS de WORKGROUPS diferentes solo pueden compartir datos
utilizando la memoria global, que es mucho mas lenta que la local, y no pueden
sincronizar su ejecución.
