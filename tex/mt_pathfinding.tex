% vim: set spell:

\section{\emph{Pathfinding} en video juegos de estrategia en tiempo real}

\emph{Pathfinding} es la búsqueda de una de la ruta de menor coste entre dos
nodos de un grafo. El coste puede tener distintas interpretaciones. Cuando se
trata de encontrar un camino en un grafo que representa un espacio físico, el
costo comúnmente representara la distancia entre los nodos o el tiempo necesario
para moverse de uno al otro, pero podrían tenerse en cuenta otras variables como
que zonas se desea evitar.

Existen un gran numero de algoritmos de \emph{Pathfinding} acorde a la gran
variedad de dominios en los que se aplica. Algunas de estos son redes,
robótica, simulación y video juegos, cada uno con sus propias restricciones.
Incluso dentro de los video juegos hay variedad en las restricciones. En los
juegos por turnos el jugador espera que la acción se vea detenida
momentáneamente mientras el sistema de inteligencia artificial calcula sus
movimientos, mientras que esto no es aceptable un juego en tiempo real. Así
mismo, la cantidad de agentes puede variar de unos pocos a cientos. Este ultimo
es el caso de los juegos de estrategia en tiempo real, también conocidos como
\emph{RTS}.

Los \emph{RTS} son video juegos donde los jugadores (ya sean humanos o
controlados por la computadora) comandan una gran cantidad de unidades que puede
variar desde docenas a miles. Por ejemplo en el titulo 'Age of empires II' de
1999 tiene un limite de 200 unidades por jugador con un máximo de 8 jugadores.
Esto propone un reto, ya que para mantener una cantidad de cuadros por segundo
de 30fps(aceptado como el mínimo de facto para una experiencia
placentera\cite{framerate06}) el algoritmo de \emph{pathfinding} no puede tardar
mas de 33ms por cada cuadro. Otra dificultad es el hardware. Si bien una
computadora dedicada para juegos suele tener mayores prestaciones que una de
oficina, no son comparables con aquellas preparadas para cálculos de altas
prestaciones. Según la encuesta de hardware de Steam\cite{steamSurvey} de julio
del 2014, solo menos del 40\% de las computadoras con la plataforma instalada
tienen por lo menos 8gb de memoria ram, y menos del 10\% tienen mas de cuatro
procesadores. 

Otro punto a tener en cuenta es que no solo basta con calcular la ruta, sino que
las unidades deben ser capases de recorrerla. Con tantos otros agentes en
movimiento es esperable que se produzca congestión y el ambiente se vuelve
altamente dinámico.
